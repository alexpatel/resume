\documentclass[]{article}
\usepackage{lmodern}
\usepackage{amssymb,amsmath}
\usepackage{ifxetex,ifluatex}
\usepackage{fixltx2e} % provides \textsubscript
\ifnum 0\ifxetex 1\fi\ifluatex 1\fi=0 % if pdftex
  \usepackage[T1]{fontenc}
  \usepackage[utf8]{inputenc}
\else % if luatex or xelatex
  \ifxetex
    \usepackage{mathspec}
    \usepackage{xltxtra,xunicode}
  \else
    \usepackage{fontspec}
  \fi
  \defaultfontfeatures{Mapping=tex-text,Scale=MatchLowercase}
  \newcommand{\euro}{€}
\fi
% use upquote if available, for straight quotes in verbatim environments
\IfFileExists{upquote.sty}{\usepackage{upquote}}{}
% use microtype if available
\IfFileExists{microtype.sty}{%
\usepackage{microtype}
\UseMicrotypeSet[protrusion]{basicmath} % disable protrusion for tt fonts
}{}
\usepackage{longtable,booktabs}
\ifxetex
  \usepackage[setpagesize=false, % page size defined by xetex
              unicode=false, % unicode breaks when used with xetex
              xetex]{hyperref}
\else
  \usepackage[unicode=true]{hyperref}
\fi
\hypersetup{breaklinks=true,
            bookmarks=true,
            pdfauthor={},
            pdftitle={},
            colorlinks=true,
            citecolor=blue,
            urlcolor=blue,
            linkcolor=magenta,
            pdfborder={0 0 0}}
\urlstyle{same}  % don't use monospace font for urls
\setlength{\parindent}{0pt}
\setlength{\parskip}{6pt plus 2pt minus 1pt}
\setlength{\emergencystretch}{3em}  % prevent overfull lines
\setcounter{secnumdepth}{0}

\date{}

\begin{document}

\section{Johnny Coder}\label{johnny-coder}

\begin{longtable}[c]{@{}ll@{}}
\toprule
1 MyAddress email@ex & ample.com\tabularnewline
MyTown 1000 @twitter & \_handle\tabularnewline
MyCountry 1800 my-ph & one-nr\tabularnewline
\bottomrule
\end{longtable}

\subsection{Education}\label{education}

2010-2014 (expected) : \textbf{PhD, Computer Science}; Awesome
University (MyTown)

\emph{Thesis title: Deep Learning Approaches to the Self-Awesomeness
Estimation Problem}

2007-2010 : \textbf{BSc, Computer Science and Electrical Engineering};
University of HomeTown (HomeTown)

\emph{Minor: Awesomeology}

\subsection{Experience}\label{experience}

\textbf{Your Most Recent Work Experience:}

Short text containing the type of work done, results obtained, lessons
learned and other remarks. Can also include lists and links:

\begin{itemize}
\item
  First item
\item
  Item with \href{http://www.example.com}{link}. Links will work both in
  the html and pdf versions.
\end{itemize}

\textbf{That Other Job You Had}

Also with a short description.

\subsection{Technical Experience}\label{technical-experience}

My Cool Side Project : For items which don't have a clear time ordering,
a definition list can be used to have named items.

\begin{itemize}
\itemsep1pt\parskip0pt\parsep0pt
\item
  These items can also contain lists, but you need to mind the
  indentation levels in the markdown source.
\item
  Second item.
\end{itemize}

Open Source : List open source contributions here, perhaps placing
emphasis on the project names, for example the \textbf{Linux Kernel},
where you implemented multithreading over a long weekend, or
\textbf{node.js} (with \href{http://nodejs.org}{link}) which was
actually totally your idea\ldots{}

Programming Languages : \textbf{first-lang:} Here, we have an
itemization, where we only want to add descriptions to the first few
items, but still want to mention some others together at the end. A
format that works well here is a description list where the first few
items have their first word emphasized, and the last item contains the
final few emphasized terms. Notice the reasonably nice page break in the
pdf version, which wouldn't happen if we generated the pdf via html.

\begin{verbatim}
: **second-lang:** Description of your experience with second-lang,
perhaps again including a [link] [ref], this time placing the url
reference elsewhere in the document to reduce clutter (see source
file).

: **obscure-but-impressive-lang:** We both know this one's pushing
it.
 
: Basic knowledge of **C**, **x86 assembly**, **forth**, **Common Lisp**
\end{verbatim}

\subsection{Extra Section, Call it Whatever You
Want}\label{extra-section-call-it-whatever-you-want}

\begin{itemize}
\item
  Human Languages:
\item
  English (native speaker)
\item
  ???
\item
  This is what a nested list looks like.
\item
  Random tidbit
\item
  Other sort of impressive-sounding thing you did
\end{itemize}

\end{document}
